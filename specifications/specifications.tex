\documentclass[a4paper,11pt]{article}
\usepackage[utf8]{inputenc}
\usepackage[T1]{fontenc}
\usepackage[french]{babel}
\usepackage{makeidx}

\title{TER 2019 - Spécifications}
\author{Maxime Gonthier - Benjamin Guillot - Laureline Martin}
\begin{document}
	\pagenumbering{gobble}\clearpage
	\maketitle

\newpage
\section{Les données}
	On va donc répartir les données de la fac en 4 module\\
	\begin{itemize}
		\item Module cours
		\item Module salle de classe
		\item Module étudiant
		\item Module professeur
	\end{itemize}
	blabla...\\
\section{Contraintes}
	Pour optimiser, nous faisons face à plusieurs contraintes, toutes ne sont pas 
	de même "importance". Nous allons donc devoir définir un ordre de priorité sur 
	les contraintes, ainsi lors de l'optimisation par notre algorithme, nous 
	pourrons ajuster et obtenir de meilleurs résultats même si certaines contraintes
	"faibles" sont violées.\\
	\subsection{Hierarchie des contraintes}
		\begin{enumerate}
			\item Enfant (contraintes matin et/ou soir)
			\item Travail (autre que la fac)
			\item Salle utilisée par deux cours différents pour des horaires 
				qui se chevauchent
			\item Horaire de cours qui se chevauchent pour un étudiant
			\item autres ?
		\end{enumerate}
	Encore du blabla ...\\
	\subsection{Sélections par heuristiques}
		Les différentes heuristiques que nous allons utiliser pour jouer sur les 
		contraintes. \\
\section{Affectation}
	Je ne me souviens plus de ce qu'il faut mettre ici. \\

\end{document}