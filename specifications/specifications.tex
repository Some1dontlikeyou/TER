\documentclass[a4paper,11pt]{article}
\usepackage[utf8]{inputenc}
\usepackage[T1]{fontenc}
\usepackage[french]{babel}
\usepackage{makeidx}

\title{TER 2019 - Spécifications}
\author{Maxime Gonthier - Benjamin Guillot - Laureline Martin}
\begin{document}
	\pagenumbering{gobble}\clearpage
	\maketitle

\newpage
\section{Les données}
	On va donc répartir les données de la fac en 4 module\\
	\begin{itemize}
		\item Module cours
		\item Module salle de classe
		\item Module étudiant
		\item Module professeur
	\end{itemize}
	\subsection{Module cours}
		Le module cours est une classe. C'est elle que l'on va déplacer lors de l'affectation. Elle contient : 
		\begin{enumerate}
			\item Durée : un entier
			\item Une liste des étudiants assistant à ce cours : $$un tableau ? un char* ? un tableau regroupant les id des etudiants (tableau d'int du coup)? $$
			\item Le type de salle utilisé : un entier, 0 pour une salle de TP et 1 pour un autre salle
			\item $$Un indice de flexibilité ?  calculé en fonction de celle des etudiants ? $$
			\item Le nombres d'étudiant : un entier
			\item Un numéro de salle : un entier
			\item Le numéro du professeur donnant ce cours : un entier
			\item L'horaire de début : un float
			\item L'horaire de fin : un float
		\end{enumerate}
	\subsection{Module salle de classe}
		$$DESCRIPTION SALLE DE CLASSE$$
		\begin{enumerate}
			\item Numéro de salle : un entier
			\item Une localisation : un entier - 0 pour proche de l'arrêt, 1 pour modérément éloigné, 2 pour éloigné-. 
			\item Un type de salle, c'est à dire une salle de TP, ou une salle classique : un entier, 0 pour une salle de TP et 1 pour un autre salle
			\item Une capacité maximale : un entier
		\end{enumerate}
	\subsection{Module etudiant}
		$$DESCRIPTION ETUDIANT$$
		\begin{enumerate}
			\item Temps de trajet : un entier
			\item Distance entre son domicile et l'université : un entier, 0 si l'étudiant habite a moins de 15 min de la fac, 1 si il habite
			entre 15 et 45 min de la fac, 2 sinon
			\item $$Flexibilité ? calculé en fonction de la distance de trajet, le temps + contrainte forte pas encore défini dans les structures $$
			\end{enumerate}
	\subsection{Module professeur}
		$$DESCRIPTION PROFESSEUR$$
		\begin{enumerate}
			\item Un numéro de professeur : un entier
			\item Une plage de disponibilité : $$fourni individuellement par chaque professeur$$
			\end{enumerate}
	$$Conclure sur les liens entre chaque classes$$
\section{Contraintes}
	Pour optimiser, nous faisons face à plusieurs contraintes, toutes ne sont pas 
	de même "importance". Nous allons donc devoir définir un ordre de priorité sur 
	les contraintes, ainsi lors de l'optimisation par notre algorithme, nous 
	pourrons ajuster et obtenir de meilleurs résultats même si certaines contraintes
	"faibles" sont violées.\\
	\subsection{Contraintes dures}
		\begin{enumerate}
			\item Avoir une personne à charge ce qui impose un horaire le matin et/ou le soir.
			Exemple : Sois X l'heure de début d'un cours, si un enfant doit être déposé à l'école à 9h on a : X > 9 + (indice de distance de cet étudiant)*30 min
					  Sois Y l'heure de fin d'un cours, si un enfant doit être récupéré à l'école à 17h on a : Y < 17 - (indice de distance de cet étudiant)*30 min
			\item Avoir un travail, cela impose la même chose que la contraintes précédentes
			\item Salle utilisée par deux cours différents pour des horaires 
				qui se chevauchent
			\item Un élève qui suit deux cours dont les horaires se chevauchent
		\end{enumerate}
	\subsection{Contraintes faibles}
		\begin{enumerate}
			\item $$Jsp j'ai tout mis en contraintes durs$$
		\end{enumerate}
	$$Conclure sur les contraintes et leurs future utilisation$$\\
	\subsection{Sélections par heuristiques}
		Les différentes heuristiques que nous allons utiliser pour jouer sur les 
		contraintes. $$surement tabou$$ \\
\section{Affectation}
	On va affecter chaque cours à un horaires sans prendre en compte les transports. On ne va utiliser que les contraintes énoncés précédemment. \\
	
\section{Evaluation de chaque affectation}
	blabla \\
	
\section{Début de reflexion sur l'implémentation des transports}
	blabla \\

\end{document}
